\documentclass[a4paper]{article} %Formato

%----------------------------------------------------------
%               PAQUETES
%----------------------------------------------------------
\usepackage[utf8]{inputenc} %Para no tener problemas con ciertos caracteres
\usepackage[spanish]{babel} %Español
\usepackage{graphicx} %Insertar imágenes
\usepackage[table,xcdraw]{xcolor} %Colores
\usepackage[most]{tcolorbox} %Cajas (En www.overleaf.com/latex/examples/drawing-coloured-boxes-using-tcolorbox/pvknncpjyfbp copiamos los comandos de la caja que queremos y luego lo editamos)
\usepackage[margin=2cm,top=2cm, includefoot]{geometry} %Márgenes
\usepackage{fancyhdr} %Definir estilo de página
\usepackage{smartdiagram} %Para insertar diagrams (En www.javatpoint.com/latex-smart-diagrams copiamos los comandos del estilo de diagrama que queremos y luego lo editamos)
\usepackage{listings} %Para insertarcode (En www.overleaf.com/learn/latex/Code_listing copiamos los comandos del estilo de código que queremos y luego lo editamos)
\usepackage{zed-csp} %Para inserta esquemas (En https://ctan.math.illinois.edu/macros/latex/contrib/zed-csp/zed2e.pdf podemos ver los distintos estilos de esquema)
\usepackage{parskip} %Elimina la sangría
\usepackage[hidelinks]{hyperref} %Gestión de hipervínculos
\usepackage{float}
\usepackage{amsmath} %Formulas matemáticas

%----------------------------------------------------------
%               DEFINICIÓN DE VARIABLES
%----------------------------------------------------------
\pagenumbering{arabic} %Numeración página
\cfoot{\thepage} %Numeración página
\newcommand{\HRule}{\rule{\linewidth}{0.5mm}} % Defines a new command for the horizontal lines, change thickness here
%\newcommand{\company_logo}{Images\company_logo.png} %Company logo

%----------------------------------------------------------
%               DEFINICIÓN DE COLORES
%----------------------------------------------------------
%Colores predeterminados:
%black, blue, brown, cyan, darkgray, gray, green, lightgray, lime, magenta, olive, orange, pink, purple, red, teal, violet, white, yellow.

%Colores personalizados:
\definecolor{Ejemplo}{HTML}{69A84F}

%----------------------------------------------------------
%               DOCUMENTO
%----------------------------------------------------------
\begin{document}

%----------------------------------------------------------
%               PORTADA
%----------------------------------------------------------
    \begin{titlepage}
        \centering
%-----------Cabecera--------------------
        \textsc{\LARGE COMPAÑÍA}\\[1.5cm] % Nombre de tu compañia
        \includegraphics[width=0.6\textwidth]{company_logo.png}\par\vspace{1cm} %Logo de tu compañia
        \textsc{\Large Texto}\\[0.5cm] %Texto mayor
        \textsc{\large Texto}\\[0.5cm] %Texto menor 
%-----------Titulo--------------------
        \HRule \\[0.4cm]
        {\huge \bfseries Título}\\[0.4cm] %Titulo de tu documento
        \HRule \\[1.5cm]
%-----------Autor--------------------
        \begin{minipage}{0.75\textwidth}
        \begin{flushleft} \large
            \centering
            Nombre \textsc{Apellidos}\\ %Tu nombre
        \end{flushleft}
        \end{minipage}\\[2cm]
%-----------Fecha--------------------
        {\large \today}\\[2cm] %Fecha, cambia \today para establecer una fecha precisa
        \vfill %Rellena el resto de la página con espacio en blanco
        
    \clearpage
    \end{titlepage}

%----------------------------------------------------------
%               CABEZERA
%----------------------------------------------------------
    \setlength{\headheight}{41.75pt}
	
    \pagestyle{fancy}
    \fancyhf
    \fancyhead{\ }
    \lhead{\includegraphics[width=5cm]{company_logo.png}} %%Parte izquierda encabezado
    \rhead{Título\\} %Parte derecha encabezado
    \renewcommand{\headrulewidth}{3pt} %Grosor línea encabezado

%----------------------------------------------------------
%               Tabla de Contenidos
%----------------------------------------------------------
    %Si no aparece el índice en español descomentamos el siguiente comando:
    %\addto\captionsspanish{\renewcommand{\contentsname}{Índice}}
    \tableofcontents
    \clearpage

%----------------------------------------------------------
%               CUERPO
%----------------------------------------------------------
    \section{Ejemplo}
    Ejemplo de texto
        \subsection{Ejemplo de subsección}
        Hola



\end{document}